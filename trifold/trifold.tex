% --------------------------------------------------------
% ----------------------PACKAGES--------------------------
% --------------------------------------------------------
\documentclass{article}
% \usepackage{showframe}
% \usepackage{fancyhdr}
\usepackage{enumitem} % noitemsep
\usepackage{amsmath, amsfonts, amsthm, amssymb, algorithm, graphicx, mathtools,xfrac}
\usepackage{listings}
\usepackage{titling}
% hyperlinks
\usepackage{hyperref}
\usepackage{xcolor}
\hypersetup{
    colorlinks=true,
    linkcolor={red!75!black},
    filecolor=magenta,
    urlcolor=blue,
}
\usepackage[a4paper,margin=1cm,landscape]{geometry}
\usepackage{tikz} % drawing
% \usepackage{fullpage}
\usepackage[UKenglish]{isodate}% http://ctan.org/pkg/isodate
\usepackage{fp}
\usepackage{array}
\usepackage{etoolbox} %\ifstrequal



\newtheorem{theorem}{Theorem}
\newtheorem{lemma}[theorem]{Lemma}
\newtheorem{invariant}[theorem]{Invariant}
\newtheorem{corollary}[theorem]{Corollary}
\newtheorem{definition}[theorem]{Definition}
\newtheorem{property}[theorem]{Property}
\newtheorem{proposition}[theorem]{Proposition}

% --------------------------------------------------------
% ---------------------NEW COMMANDS-----------------------
% -------------Automatic /label for a section-------------
% --------------------------------------------------------

% usage: same as section, subsection, etc -- automatically adds label.
\newcommand{\autosection}[1]{\section{#1}\label{#1}}
\newcommand{\autosubsection}[1]{\subsection{#1}\label{#1}}
\newcommand{\autosubsubsection}[1]{\subsubsection{#1}\label{#1}}

% usage: \refer{name} creates ``name (see \S\ref{name})''. Place \refer{[...]} around your content to auto-link it.
\newcommand{\refer}[1]{#1 (see \S \ref{#1})}

% usage: \centerimage{img.extension} creates a scaled, centered image in the document
\newcommand{\centerimage}[1]{ \begin{center}\includegraphics[width=1.0\textwidth]{#1}\par\end{center}}

% --------------------------------------------------------
% -----------FORMATTING OF DOCUMENT, VARIABLES------------
% --------------------------------------------------------

\newcommand\foldwidth{\textwidth/3}
\newcommand\secondfold{\foldwidth * 2}
\newcommand{\commenttext}[1]{\em\scriptsize{#1}}
\newcommand{\emphtext}[1]{\textbf{\textit{\Large{#1}}}}
\newcommand{\superemphtext}[1]{\textbf{\textit{\Huge{#1}}}}


% --------------------------------------------------------
% --------------------FORMAT COMMANDS---------------------
% --------------------------------------------------------
\newcommand\pageoutline{
% \draw (0,0) -- (\textwidth, 0);
\draw[dashed] ({\foldwidth - .5cm},0) -- (\foldwidth-.5cm, \textheight);
% \draw[dashed] (\foldwidth,0) -- (\foldwidth, \textheight);
\draw[dashed] (\foldwidth * 2 + .5cm,0) -- (\foldwidth * 2 + .5cm, \textheight);
% \draw (0,\textheight) -- (\textwidth, \textheight);
}



\newcommand{\abilityscore}[3]{
% #1: vertical align
% #2: Name
% #3: list of modified stats
\draw ({\secondfold + \foldwidth/8}, #1) -- node[above] {\commenttext{score}}
      ({\secondfold + \foldwidth/8 * 2}, #1) --
      ({\secondfold + \foldwidth/8 * 2.25}, {#1 - \textheight/40}) --
      ({\secondfold + \foldwidth/8 * 2.5}, #1) -- node[above] {\commenttext{mod}}
      ({\secondfold + \foldwidth/8 * 3.5}, #1) -- node[right] {\bf\Large{#2}}
      ({\secondfold + \foldwidth/8 * 3.5}, {#1 - \textheight/14}) --
      ({\secondfold + \foldwidth/8 * 2.5}, {#1 - \textheight/14}) --
      ({\secondfold + \foldwidth/8 * 2.25}, {#1 - \textheight/14 + \textheight/40}) --
      ({\secondfold + \foldwidth/8 * 2}, {#1 - \textheight/14}) --
      ({\secondfold + \foldwidth/8}, {#1 - \textheight/14}) -- cycle;
\node[align=right,left] at ({\secondfold + \foldwidth}, {#1 - \textheight/28}) {#3};
\draw[very thin] ({\secondfold + \foldwidth/8}, {#1 - \textheight/14}) -- ({\secondfold + \foldwidth/8 * 2}, #1);
}
\newcommand\abilitysep{\textheight/8}



\newcommand\inputboxwidth{\foldwidth/6}
\newcommand\inputboxheight{\foldwidth/12}
% \newcommand{\inputbox}[3]{
% \draw ({#2 - \inputboxwidth/2}, {#3 - \inputboxheight/2}) rectangle
%       ({#2 + \inputboxwidth/2}, {#3 + \inputboxheight/2}) node[pos=0.5,yshift=-\inputboxheight/2,below] {\commenttext{#1}};
% }
\newcommand{\inputbox}[4]{
% #1: under text
% #2: left text
% #3: x center pos
% #4: y center pos
\draw ({#3 - \inputboxwidth/2}, {#4 - \inputboxheight/2}) rectangle
      ({#3 + \inputboxwidth/2}, {#4 + \inputboxheight/2})
      node[pos=0.5,yshift=-\inputboxheight/2,below] {\commenttext{#1}}
      node[pos=0.5,xshift=-\inputboxwidth/2,left,align=right] {\emphtext{#2}};
}
\newcommand{\boxset}[5]{
% #1: y pos
% #2: left text
% #3: first box text
% #4: second box text
% #5: third box text
\inputbox{#3}{#2}{\foldwidth/5 * 2}{#1}
\inputbox{#4}{}{\foldwidth/5 * 3}{#1}
\inputbox{#5}{}{\foldwidth/5 * 4}{#1}
}

\newcommand{\inputline}[1]{
    $\underset{\text{\commenttext{#1}}}{\underline{\hspace{1cm}}}$
}

\newcommand{\longinputline}[1]{
    $\underset{\text{\commenttext{#1}}}{\underline{\hspace{1.5cm}}}$
}


\newcommand\acradius{\foldwidth/16}
\newcommand\acy{\abilitysep * 5}

\newcommand\linesep{\foldwidth/14}
\newcommand\maxlinenum{10}
\newcommand{\schartvline}[1]{
    \draw[thin] ({\foldwidth - \linesep * (\maxlinenum - #1 + 1) - .5cm}, 2\abilitysep/3) -- ({\foldwidth - \linesep * (\maxlinenum - #1 + 1) - .5cm},\textheight - \abilitysep) node[above] {\commenttext{+#1}};
}

\newcommand\schartnumlines{26}
\newcommand\minschartx{\foldwidth - \linesep * \maxlinenum - .5cm}
\newcommand\maxschartx{\foldwidth - \linesep * \maxlinenum - 9 - .5cm}
\newcommand{\schartstepx}{(\textheight-\abilitysep - 2\abilitysep/3)/\schartnumlines}
\newcommand{\scharthline}[1]{
    \draw[dotted,thin] (\minschartx, {\textheight-\abilitysep - (#1 * \schartstepx)}) node[left] {\rule[0cm]{1.9cm}{.15mm}} -- ({\foldwidth - \linesep * (\maxlinenum - 9) - .5cm}, {\textheight-\abilitysep - (#1 * \schartstepx)});
}

\newcommand\numitemchartlines{26}
\newcommand\minitemchartx{2\foldwidth + 1.5cm}
\newcommand\maxitemchartx{3\foldwidth }
\newcommand{\itemchartstepx}{(\textheight-\abilitysep - 2\abilitysep/3)/\numitemchartlines}
\newcommand{\itemcharthline}[1]{
    \draw (\minitemchartx, {\textheight-\abilitysep - (#1 * \itemchartstepx)}) -- (\maxitemchartx, {\textheight-\abilitysep - (#1 * \itemchartstepx)});
}

\newcommand{\classheader}[2]{
% 1: class name
% 2: class desc
\node at (1.5\foldwidth, 19\textheight/20) {\superemphtext{#1}};
\node[text width=\foldwidth-1cm,below] at (1.5\foldwidth, 18.25\textheight/20) {
    #2};
}

% --------------------------------------------------------
% ----------------SETUP / DOC PROPERTIES-----------------
% --------------------------------------------------------

\pagestyle{empty}

\title{}
\author{Isabel Trahan}
\date{\today }

% \lhead{\thedate}
% \chead{\thetitle}
% \rhead{\theauthor}

% --------------------------------------------------------
% ------------------------DOCUMENT------------------------
% --------------------------------------------------------

\begin{document}
% for each class
\foreach \classname in {wizard,cleric,warrior,thief,dwarf,elf,halfling}{

\noindent
\begin{tikzpicture}
    % outline
    \pageoutline

    % --------------------------------------------------------
    % ------------------------MAIN PAGE-----------------------
    % --------------------------------------------------------

    % ability scores
    \abilityscore{\abilitysep * 6}{STR}{
        \commenttext{Melee to-hit} \\
        \commenttext{Melee damage}
    }
    \abilityscore{\abilitysep * 5}{AGI}{
        \commenttext{Missile to-hit} \\
        \commenttext{Armor class} \\
        \commenttext{Reflex saves} \\
        \commenttext{Initiative}
    }
    \abilityscore{\abilitysep * 4}{STA}{
        \commenttext{Hit points} \\
        \commenttext{Fortitude saves}
    }
    \abilityscore{\abilitysep * 3}{PER}{
        \commenttext{Willpower saves}
    }
    \abilityscore{\abilitysep * 2}{INT}{
        \commenttext{Languages}
    }
    \abilityscore{\abilitysep * 1}{LUCK}{
        \commenttext{Crit/fumble rolls} \\
        \commenttext{Ability burn (+1/pt)} \\
        \rule[-.3cm]{2cm}{.15mm} % write-in
    }
     % revision date
    \node[above] at ({\secondfold + 1/2 * \foldwidth}, 0) {{\tiny Revised \cleanlookdateon\today}};

    % Title + name
    \draw ({\secondfold + \foldwidth/8}, {\abilitysep * 8}) rectangle ({\secondfold + \foldwidth - \abilitysep * .5}, {\abilitysep * 7})
    node[pos=.5,anchor=center,fill=white,draw,yshift= -\abilitysep * .75, xshift=-\foldwidth/10] {
        \begin{tabular}{r | l}
            \emphtext{Title} & \emphtext{Name} \\ \hline
            \commenttext{Alignment:} & \rule[-.7mm]{1.5cm}{.15mm} \\ % write-in
            \commenttext{Occupation:} & \rule[-.7mm]{1.5cm}{.15mm} \\ % write-in
        \end{tabular}
    };

    \draw[very thick] ({\secondfold + \foldwidth/8 + \foldwidth/16}, {\abilitysep * 7 + \abilitysep * .2}) --
                      ({\secondfold + \foldwidth/8 + \foldwidth/4}, {\abilitysep * 7 + \abilitysep * .2});
    \draw[very thick] ({\secondfold + \foldwidth/8 + \foldwidth/26 + \foldwidth/4}, {\abilitysep * 7 + \abilitysep * .2}) --
                      ({\secondfold + \foldwidth/8 + \foldwidth/26 + \foldwidth/4 + \foldwidth/3.5}, {\abilitysep * 7 + \abilitysep * .2});

    % level circle
    \draw[fill=white] ({\secondfold - \abilitysep * .5 + \foldwidth}, {\abilitysep * 7.5}) circle (\abilitysep * .5) node[fill=white,draw,yshift= -\abilitysep * .75] {
        \begin{tabular}{r | l}
            \multicolumn{2}{c}{\emphtext{Level}} \\ \hline
            \commenttext{XP:} & \rule[-.7mm]{1cm}{.15mm} \\ % write-in
            \commenttext{Next:} & \rule[-.7mm]{1cm}{.15mm} \\ % write-in
        \end{tabular}
    };


    % --------------------------------------------------------
    % ------------------------QUICK REF-----------------------
    % --------------------------------------------------------

    \node[draw, text width=\textheight - 2.4mm, rotate=-90,right] at ({2\foldwidth - .1cm}, \textheight) {
        \emphtext{Dice Chain:}
        $d3  \rightarrow  d4  \rightarrow  d5  \rightarrow  d6  \rightarrow  d7  \rightarrow  d8  \rightarrow  d10 -– d12  \rightarrow  d14 \rightarrow d16 \rightarrow d20 \rightarrow d24^1 \rightarrow d30^2$ \\
        \commenttext{$^1$: 1d12 + (1d2-1 * 12), $^2$: 1d10 + (1d3-1 * 10)}
    };

    % \node[text width=\foldwidth-1cm,below] at (1.5\foldwidth - .5cm, \textheight) {
    %     \emphtext{Actions:}
    %     Each round, a character may move its normal speed {\em and} do one thing for each of its action dice. The actions a character may take depends on their class:
    %     \begin{itemize}[label=]
    %         \item All characters can take another movement for their actions.
    %         \item A warrior can make an attack or mighty deed of arms for each of his actions.
    %         \item A wizard can attack {\em or} cast a spell with their first die and can only cast a spell with the second die.
    %         \item An elf can attack or cast a spell with any action die.
    %         \item Most simple actions (drawing weapons, opening doors, reading scrolls) take a single action.
    %     \end{itemize}
    % };

    \draw (\foldwidth, \textheight - \abilitysep/2) -- (2\foldwidth - 1.2cm, \textheight - \abilitysep/2)
    node[pos=.5,above] {
        \expandafter\ifstrequal\expandafter{\classname}{wizard}{\emphtext{Grimorie}}{
            \expandafter\ifstrequal\expandafter{\classname}{elf}{\emphtext{Grimorie}}{
                \expandafter\ifstrequal\expandafter{\classname}{cleric}{\emphtext{Notes}}{
                    \expandafter\ifstrequal\expandafter{\classname}{dwarf}{\emphtext{Notes}}{
                        \expandafter\ifstrequal\expandafter{\classname}{thief}{\emphtext{Notes}}{
                            \expandafter\ifstrequal\expandafter{\classname}{warrior}{\emphtext{Notes}}{
                                \expandafter\ifstrequal\expandafter{\classname}{halfling}{\emphtext{Notes}}{
                                    CLASS ERROR
                                }
                            }
                        }
                    }
                }
            }
        }
    }
    -- (2\foldwidth - 1.2cm, 0) -- (\foldwidth, 0) -- cycle;

    % --------------------------------------------------------
    % -----------------------COMBAT STATS---------------------
    % --------------------------------------------------------

    \boxset{\abilitysep * 1}{Crits}{Crit die}{Crit table/pg}{Fumble die}
    \boxset{\abilitysep * 2}{Saves}{Fortitude}{Reflex}{Willpower}
    \boxset{\abilitysep * 3}{HP}{Current}{Maximum}{Hit die}
    \boxset{\abilitysep * 4}{Attack}{Attack Bonus}{Action Dice}{Initiative}

    % AC
    \draw[fill=white] (\acradius, \acy) circle (\acradius)
    % node[xshift=\acradius,right,yshift=-1.5mm] { = 10 $+$ \inputline{Agility} + \inputline{Armor}+ \inputline{Other} }
    node[below, draw, fill=white, yshift=-\acradius/2] { \em\textbf{AC} };
    \boxset{\acy}{: 10 -}{Agility}{Armor}{Other}
    \node[draw, fill=white] at ({(\foldwidth/5 * 1 + \foldwidth/5 * 2)/2}, \abilitysep * 5) {+};
    \node[draw, fill=white] at ({(\foldwidth/5 * 2 + \foldwidth/5 * 3)/2}, \abilitysep * 5) {+};
    \node[draw, fill=white] at ({(\foldwidth/5 * 3 + \foldwidth/5 * 4)/2}, \abilitysep * 5) {+};

    % \begin{tabular}{c c c c c c c c c}
    %     10 & + & \writein & + & \writein & + & \writein & + & \writein \\
    %        &   &  \commenttext{AGI}     &   &    \commenttext{armor} &   &   \commenttext{shield} &   & \commenttext{other} \\
    % \end{tabular}

    % equipment
    \draw (0,\textheight - \abilitysep/2) rectangle ((\foldwidth/5 * 4 + \inputboxwidth/2, \abilitysep * 6)
    node[pos=.5,yshift=(\textheight - \abilitysep/2 - \abilitysep * 6)/2,above] {\emphtext{Equipment / Stats}};

    % \node (\foldwidth/2, \abilitysep * 6);


\end{tikzpicture}
\begin{tikzpicture}
    % outline
    \pageoutline

    % --------------------------------------------------------
    % --------------------------SKILLS------------------------
    % --------------------------------------------------------
    \node at (\foldwidth/2 - .5cm, \textheight - \abilitysep/2) {\emphtext{Skills}};

    \foreach \n in {1,...,\maxlinenum}{
        \schartvline{\n}
    }

    \foreach \n in {1,...,\schartnumlines}{
        \scharthline{\n}
    }

    \inputbox{Penalty}{}{\foldwidth/2-.5cm}{\abilitysep/3}

    % --------------------------------------------------------
    % --------------------------CLASS-------------------------
    % --------------------------------------------------------

    \input{inserts/\classname.tex}
    % \node[rotate=-90,gray] at (3\foldwidth/2, \textheight/2) {\emphtext{Class insert goes here \classname}};

    % --------------------------------------------------------
    % ------------------------EQUIPMENT-----------------------
    % --------------------------------------------------------

    \node at (3\foldwidth - \foldwidth/2 + .5cm, \textheight - \abilitysep/2) {\emphtext{Possessions}};

    % \newcommand\maxitemlinenum{10}
    % \newcommand{\itemchartvline}[1]{
    %     \draw[gray] ({3\foldwidth - \linesep * (\maxlinenum - #1 + 1) - .5cm}, \abilitysep/2) -- ({3\foldwidth - \linesep * (\maxlinenum - #1 + 1) - .5cm},\textheight - \abilitysep) node[above] {\commenttext{+#1}};
    % }
    %
    % \foreach \n in {1,...,\maxlinenum}{
    %     \itemchartvline{\n}
    % }


    \foreach \n in {1,...,\numitemchartlines}{
        \itemcharthline{\n}
    }

    \draw (\maxitemchartx-1cm, 2\abilitysep/3) -- (\maxitemchartx-1cm,\textheight-\abilitysep);
    \draw (\maxitemchartx, 2\abilitysep/3) -- (\maxitemchartx, \textheight-\abilitysep);
    \node[above] at ({(\minitemchartx + \maxitemchartx-1cm)/2}, \textheight-\abilitysep) {\commenttext{Item}};
    \node[above] at ({(\maxitemchartx + \maxitemchartx-1cm)/2}, \textheight-\abilitysep) {\commenttext{\#}};

    \inputbox{cp}{}{2\foldwidth + \foldwidth/6 * 2}{\abilitysep/3}
    \inputbox{sp}{}{2\foldwidth + \foldwidth/6 * 3}{\abilitysep/3}
    \inputbox{gp}{}{2\foldwidth + \foldwidth/6 * 4}{\abilitysep/3}
    \inputbox{ep}{}{2\foldwidth + \foldwidth/6 * 5}{\abilitysep/3}

\end{tikzpicture}
}
\end{document}
