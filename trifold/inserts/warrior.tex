\classheader{Warrior}{
    \textbf{Luck}: A warrior’s Luck modifier applies to attack rolls with one specific kind of weapon, chosen at first level:\longinputline{weapon} \\
    \textbf{Weapon training}: A warrior is trained in the use of the battleaxe, club, crossbow, dagger, dart, handaxe, javelin, longbow, longsword, mace, polearm, shortbow, short sword, sling, spear, staff, two-handed sword, and warhammer. Warriors wear whatever armor they can afford. \\
    \textbf{Attack modifier}: Warriors use a ``deed die'' instead of an attack modifier. The warrior rolls this die on each attack and applies it to both his chance to hit and his damage. \\
    \textbf{Mighty deed of arms}: Prior to any attack roll, a warrior can declare a Mighty ``Deed of Arms.'' This deed is a dramatic combat maneuver within the scope of the current combat. Such maneuvers may include:
    \begin{itemize}[noitemsep]
         \item blinding
         \item disarming
         \item pushbacks
         \item trips and throws
         \item precision shots
         \item rallying allies
         \item defensive maneuvers
     \end{itemize}
    The deed does not increase damage. The warrior’s deed die (above) determines the deed’s success. If the deed die is a 3 or higher, and the attack lands, the deed succeeds. A higher roll on the deed die gives a better result (see page 88). If the deed die is a 2 or less, or the overall attack fails, the deed fails as well. \\
    \textbf{Initiative}: A warrior adds his class level to his initiative rolls.\\
    \noindent\begin{tabular}{@{}m{3.5cm} m{3cm}}
            \textbf{Crit Range}: A
            warrior's crit range
            changes by level:
        &
        \begin{tabular}{|r | l|}
            \hline
            1-4 & 19-20 \\
            5-8 & 18-20 \\
            6-10 & 17-20 \\
            \hline
        \end{tabular}
    \end{tabular}
}
