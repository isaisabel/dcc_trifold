\classheader{Thief}{
    \textbf{Skills}: Thieves receive bonuses to various skills according to their alignment. See table 1-9. \\
    \textbf{Thieves’ Cant}: Thieves speak a secret language called the cant, which can be used to communicate with other thieves covertly through double-entendre. \\
    \textbf{Luck}: For each point of Luck expended, the thief rolls \inputline{die} and applies that modifier to his roll. Thieves restore spent luck upon sleeping at a rate equal to their level. \\
    \textbf{Weapon training}: A thief is trained the use of the blackjack, blowgun, crossbow, dagger, dart, garrote, longsword, short sword, sling, and staff. Thieves are careful in their choice of armor, as it affects the use of their skills. \\
    \textbf{Sneak}: A thief's sneak check is never opposed. The hard DCs are noted as follows:
        \begin{itemize}[noitemsep]
            \item cushioned surfaces (grass): \underline{DC 5}.
            \item stone surfaces: \underline{DC 10}.
            \item moderately noisy surfaces (creaky boards): \underline{DC 15}.
            \item extremely noisy surfaces (water, leaves): \underline{DC 20}.
        \end{itemize}
    \textbf{Hide}: A thief's hide check is never opposed. The hard DCs are noted as follows:
        \begin{itemize}[noitemsep]
            \item at night or in dim lighting: \underline{DC 5}.
            \item under a full moon: \underline{DC 10}.
            \item in daylight but in a dark shadow or behind an object: \underline{DC 15}.
            \item in broad daylight with minimal obstruction: \underline{DC 20}.
        \end{itemize}
    \textbf{Backstab}: When attacking a target from behind or when the target is otherwise unaware, the thief makes the check with a bonus according to their backstab skill. In addition, if they hit, the thief automatically achieves a critical hit, rolling on the crit table as per his level. Backstab attempts can only be made against creatures with clear anatomical vulnerabilities.
}
