\classheader{Elf}{
    \textbf{Luck}: At 1st level, an elf may choose to apply their luck modifier to one spell of their choosing. This modifier does not change as the elf's luck score changes \\
    \textbf{Weapon training}: An elf is trained to use the dagger, javelin, lance, longbow, longsword, shortbow, shortsword, staff, spear and two-handed sword. Elves often wear armor, although it does affect their spellcasting. \\
    \textbf{Magic}: Elves cast spells by making a spell check. A elf's spell check is usually 1d20 + intelligence modifier + level. Elves automatically receive the spells \textit{invoke patron} and \textit{patron bond}. \\
    % \textbf{Mercurial Magic}: The effect of a magical spell varies according to who casts it. When you learn a new spell, roll on table 5-2 (pg 111) and adjust by your luck modifier x 10\%. \\
    \textbf{Corruption}: Magical corruption results from natural 1s on spell checks, such as misfire and/or corruption. Corruption effects are permanent but modified by the wizard's luck score.\\
    \textbf{Infravision}: An elf can see in the dark up to 60'. \\
    \textbf{Immunities}: Elves are immune to magical sleep and paralysis. \\
    \textbf{Vulnerabilities}: Elves are extremely sensitive to the touch of iron and iron alloys. Direct contact over prolonged periods causes burning and exposure at a distance makes them uncomfortable. An elf may not wear iron armor or use iron weapons for extended periods. Prolonged contact causes 1hp of damage per day of direct contact.\\
    \textbf{Hightened Senses}: All elf characters receive a +4 bonus to detecting secret doors. When passing within 10 feet of a secret door, elves are entitled to a check to detect it.
}
\draw (1.1\foldwidth, .19\textheight) rectangle (1.9\foldwidth, .13\foldwidth) node[pos=.5,yshift=.072\textheight, above] {\textbf{Corruption effects}};
